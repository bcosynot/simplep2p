\documentclass[12pt,twosided,a4paper]{article}

\usepackage{tabulary}
\usepackage[utf8]{inputenc}
\usepackage[T1]{fontenc}
%\usepackage[urw-garamond]{mathdesign}
\usepackage{garamondx}
%\usepackage{gfsartemisia}
%\usepackage[T1]{fontenc} % Use 8-bit encoding that has 256 glyphs
%\usepackage{garamondx}
%\usepackage[garamondx,cmbraces]{newtxmath}
%\usepackage{fourier} % Use the Adobe Utopia font for the document - comment this line to return to the LaTeX default
%\usepackage[utf8]{fontenc}
\usepackage[english]{babel}
\usepackage{amsmath,amsfonts,amsthm,amssymb}

\usepackage[margin=0.8in]{geometry}

\numberwithin{equation}{section} % Number equations within sections (i.e. 1.1, 1.2, 2.1, 2.2 instead of 1, 2, 3, 4)
\numberwithin{figure}{section} % Number figures within sections (i.e. 1.1, 1.2, 2.1, 2.2 instead of 1, 2, 3, 4)
\numberwithin{table}{section} % Number tables within sections (i.e. 1.1, 1.2, 2.1, 2.2 instead of 1, 2, 3, 4)

\usepackage{fancyhdr} % Custom headers and footers
\pagestyle{fancyplain} % Makes all pages in the document conform to the custom headers and footers
\fancyhead[L]{CSci 5211}
\fancyhead[C]{Fall 2013 Programming Project}
\fancyhead[R]{Vivek Ranjan - 4924202}
\fancyfoot[L]{} % Empty left footer
\fancyfoot[R]{} % Empty center footer
\fancyfoot[C]{\thepage} % Page numbering for right footer

\usepackage{listings}
\usepackage{hyperref}

\title{
\Huge \textbf{Fall 2013 Programming Project}\\[3pt]
\large \textbf{CSci 5211}
}

\author{Vivek Ranjan - 4924202}
\date{Due on: 10\textsuperscript{th} December, 2013}
\begin{document}
	\maketitle
\thispagestyle{empty}


\tableofcontents
\newpage

\section{About}
The project is a Peer-to-Peer (P2P) file sharing system, written in Java and
implemented by using sockets. This document lists down all the features that have been
implemented and how to use the various programs.

\section{Archive Contents}
The archive contains the following folders/directories:
\begin{description}
\item[code]	Contains all the source files for the project. All source files are
heavily commented to ensure high readability and make the code easy to
understand for non-java programmers too.
\item[lib] Contain a few libraries (mostly related to logging) that have been
used in the programs.
\item[docs] Contains JavaDocs detailing all the classes and methods that are
used. Open index.html in any browser to start.
\item[executables] Contains all the executable files that can be used to run the
programs. Instructions on how to run the programs are provided below. These are
independent (do not require any of the other files in the archive) and portable
(can be run on any environment that satisfies the requirements stated below.)
\end{description}

\section{Running The Programs}
\subsection{Requirements}
The programs do require java to be installed on
the machine prior to running it. Instructions for installing the Java Runtime
Environment (JRE) can be found on the following link -
\url{https://www.java.com/en/download/help/download_options.xml}. 
Do note that the CS grad lab machines (cello, trombone, etc) and the CSE
lab machines, both already have Java installed and can be used to run these
programs. These programs have been tested to run on the Linux machines on both
labs.
\newline
Running the programs is simple and follows the requirements stated in the
programming project description. 
\subsection{Central Server}
The central server program can be run using the \textbf{centralserver.jar} file
in the executables directory.
The following command should be used:
\begin{lstlisting}
java -jar centralserver.jar
\end{lstlisting}
The central server program can also take two optional command line arguments.
Namely:
\begin{description}
\item[-f<path/to/file>] Stores the list of peers in the specified file. Default
file is peerList.txt in the current working directory. For example:
\begin{lstlisting}
java -jar centralserver.jar -fpeers.txt
\end{lstlisting}
\item[-p<port>] Specifies what port the program should use to listen for
incoming messages. Default port used by the program is $5555$. For example:
\begin{lstlisting}
java -jar centralserver.jar -p5355
\end{lstlisting}
\end{description}

\subsection{Peer}
The peer program can be run using the \textbf{peer.jar} file in the executables
directory.
The following command should be used:
\begin{lstlisting}
java -jar peer.jar <path/to/working-directory>
<central-server-ip>:<central-server-port>
\end{lstlisting}
Example:
\begin{lstlisting}
java -jar peer.jar peer3 kh2170-01.cselabs.umn.edu:5555
\end{lstlisting}
The first argument is the \emph{full} path to the working directory that will be
used to share files to and from the peers on the network.
The second argument is the complete address of the central server, including the
port, in the IP:port format. The IP can be the numeric IP or the host address of
the central server. Both will work. The peer program does not support any
additional arguments.

\section{Using the Peer program}
The peer program is completely multi-threaded and any number of the mentioned
tasks can be run simultaneously. It supports the following functionalities:
\begin{description}
	\item[get] Searches the network for the specified file and downloads it into
	the working directory.
	\item[share] Sends the specified file to the specified peer.
	\item[list] Lists all the files in the working directory.
	\item[quit] Terminates the program after informing its neighbours and the
	central server about the same.
\end{description}
The usage of each command is explained below.
\subsection{get}
The \textbf{get} command is used to download a file from the P2P network after
searching for it. The file is searched by flooding all the neighbours of the requesting
peer with a lookup command, who in turn flood their neighbours with the same.
This continues till the file is found and a direct connection is established
between the source and destination peer hosts to download the file. The syntax
for the command is:
\begin{lstlisting}
get <filename>
\end{lstlisting}
Example:
\begin{lstlisting}
get mydoc4.pdf
\end{lstlisting}
The file is downloaded and stored in the working directory specified while
initialising the peer program.

If the file is not found within 5 seconds, a timeout error message will be
displayed and the peer will stop waiting for the file.

\subsection{share}
The \textbf{share} command is used to send a file to a peer host in the network,
that may or may not be the source peer host's neighbour. The syntax for this
command is:
\begin{lstlisting}
share <filename> <ip>:<port>
\end{lstlisting}
Example:
\begin{lstlisting}
share mydoc7.psd trombone.cs.umn.edu:55932
\end{lstlisting}
If an invalid filename is provided, command will be terminated after displaying
an error message. The port number is the port that the destination peer is using
to listen for messages.


\subsection{list}
This command just lists down all the files in the working directory. It does not
take or need any arguments

\subsection{quit}
This command terminates the peer and exits the program. Before completely
exiting the program, it informs the neighbours and the central server about the
same.
\section{Further Information}
A quick bird's eye view of the inner working of the programs can be achieved by
taking a look at the JavaDocs available in the \textbf{docs} directory. It contains all the classes
and methods used in the program.

Further exploration can be done by going through the source code in the
\textbf{code} directory. All source files have been properly and thoroughly
commented so that it is easy to understand the code even for a non-java
programmer.
\end{document}